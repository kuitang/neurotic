\documentclass{article}

\begin{document}
\title{Weekly Paragraphs}
\author{Kui Tang}
\date{\today}
\maketitle
\section*{October 3, 2012}
To appear in: Related Work

All known connectomes reply on manual labels. White et al. published the first connectome of \emph{C. elegans} in 1986, marking axons, dendrites, and cell bodies in each frame and tracing the labels through frames \cite{White12111986}. More recent work uses computer assistance but fundamentally rely on a skilled human's identification of neural bodies \cite{Jarrell2012,Bock2011Roberts2011}.

Koshevoy et al. \cite{Koshevoy2006} use a variant of SIFT \cite{Lowe2004} to register a stack of images into a 3D volume. According to Koshevoy, images between images exhibit warping due to the physical slicing, physical structural changes, and a distinct rotation and displacement on each frame. From the displacements of individual keypoints, a global transformation vector is calculated, allowing slices to be properly aligned. However, the SIFT features do not capture neural processes well, and thus is unsuited for identifying neurons across images.

\bibliographystyle{plain}
\bibliography{refs.bib}
\end{document}
