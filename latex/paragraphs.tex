\documentclass{article}

\begin{document}
\title{Weekly Paragraphs}
\author{Kui Tang}
\date{\today}
\maketitle
\section*{October 3, 2012}
To appear in: Related Work

All known connectomes reply on manual labels. White et al. published the first connectome of \emph{C. elegans} in 1986, marking axons, dendrites, and cell bodies in each frame and tracing the labels through frames \cite{White12111986}. More recent work uses computer assistance but fundamentally rely on a skilled human's identification of neural bodies \cite{Jarrell2012,Bock2011Roberts2011}.

Koshevoy et al. \cite{Koshevoy2006} use a variant of SIFT \cite{Lowe2004} to register a stack of images into a 3D volume. According to Koshevoy, images between images exhibit warping due to the physical slicing, physical structural changes, and a distinct rotation and displacement on each frame. From the displacements of individual keypoints, a global transformation vector is calculated, allowing slices to be properly aligned. However, the SIFT features do not capture neural processes well, and thus is unsuited for identifying neurons across images.

\section*{October 15, 2012}
Introduction:

Related Work:

EM images present several challenges for classical segmentation algorithm. First, structures or interest are small and densely packed \cite{Jain2007}. Second, due to the physical cutting process, images may be warped from slice to slice \cite{Koshevoy2006}.

Semi-automated methods take small amounts of user input and then use the morphology of the image segment entire neural processes. Marker-Controlled watersheds \cite{Gonzales2008} is a classic technique, requiring the user to mark background and foreground. Roberts et al. \cite{Roberts2011} require a "scribble" inside a neuron on the first and last slices. At each slice, their algorithm segments the neuron by minimizing the anisotropic weighted total variation \cite{Unger2009} and propagating the inferred boundary as soft constraints for the next layer.

Unfortunately, current baseline speeds of computer-assisted manual tracing would take 10,000 person-years to construct a mouse cortical column, suggesting that fully automatic methods are necessary \cite{Briggman2006}.

Fully-automated methods use manual labels as input for supervised learning. Segmentation can be done in voxel space, after which the learned segmentation can be directly used as a 3D reconstruction. Because cells are usually disconnected, segmentation can be reduced to binary classification of whether voxels are inside or outside a cell. Jain et al. \cite{Jain2007} achieved an 8.0\% test error with a 34,000 convolutional neural network, a statistically significant improvement other the 18.95\% error of thresholding, while MRFs and CRFs do no better than thresholding. Andres et al. obtain similar performance, at 8\% misclassification, using the watershed algorithm over a probability map of a random forest with hand-engineered features \cite{Andres2008, Gonzales2008}. However, the dataset used by both used a special intracellular stain, making the segmentation job easier.

Alternatively, one can segment in pixel space and track the identity of segmented bodies in the vertical dimension. This is motivated by the poor vertical resolution of EM images: the thinnest slices are 40nm think \cite{Kaynig2010b, Briggman2006}. Jurrus et al. \cite{Jurrus2008} first obtain 2D segmentations with watershed after preprocessing linear diffusion filtering (NB: learn what this is!), and construct a weighted graph of these segments, with edges between adjacent sections with edge costs favoring high cross-correlation of segments low lateral displacement. Each neural process is thus a shortest path through this graph, computed by Dijkstra's algorithm. This cost function is robust to merging and loss of sections, but because Dijkstra's algorithm finds a single path, it cannot model branching. Kaynig et al. \cite{Kaynig2010a} presuppose segmentation and represent the image as a graph of regions, like Jurrus. Pairwise features are represented as edge weights, and a random forest classifier is trained to output a scalar similarity score for each pair. Then a probabilistic model encoding geometric constraints with cross-section correspondences as hidden variables is trained with EM. Finally, the maximum-likelihood result is the 3D reconstruction. These pairwise correspondences are fed to an agglomerative clustering algorithm, constructing the neural processes from ground-up in the same manner as a human. Unlike in Jurrus, processes may merge. However, each region is explicitly limited to two correspondences: one above and below, so processes still cannot branch.

Feature engineering has improved 2D segmentation. Venkataraju et al. \cite{Venkataraju2009} generated 100 Hessian neighborhood features at each pixel and classified Jurrus's dataset with AdaBoost. Lucchi et al. \cite{Lucchi2010} develop ray descriptors, which capture shape features without a predefined shape model. These features are rotation, scale, and translation invariant, but strongly discriminate the types of objects observed in EM images. There features were used in SVMs to classify interior and boundary points of mitochondria, which are considerably harder to classify than other cellular structures \cite{Kaynig2010b}, and a graph cut found the final 2D segmentation with 98\% accuracy. Kaynig et al. encode perceptual grouping constraints---rules of thumb used by human experts to group neuron structures---as pairwise interaction terms in an energy function solved by graph cut. Minimizing the energy function gives a segmentation, and identifying maximum overlapping regions between slices is sufficient to obtain a 3D reconstruction 


\bibliographystyle{plain}
\bibliography{refs.bib}
\end{document}
